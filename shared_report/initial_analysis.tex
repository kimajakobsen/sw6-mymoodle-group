\chapter{Initial Requirements}
\label{sec:initialAnalysis}
To determine how to improve Moodle to support the Aalborg PBL model we conducted two informal discussions with E-L\ae{}ringssamarbejdet ved Aalborg Universitet (ELSA)~\cite{elsa} and the Master in Problem Based Learning (MPBL) department~\cite{mpbl} at Aalborg University.
This section accounts for the results of the discussion.
Based on the results we decide how to solve the problem explained in conceptual terms.

\section{Discussion with Expert Users}  
\label{sub:expertUsers} 
We conducted meetings with ELSA and MPBL, in which we discussed what improvements we could make to Moodle in regards to the Aalborg PBL model.

\subsection{ELSA}
\label{sub:elsaInterview}
\begin{itemize}
	\item Type: Informal meeting
	\item Participants: Lillian Buus~\cite{lillian}, Marie Glasemann~\cite{marie}, Mads Peter Bach~\cite{mads}, and us 
	\item Date: February 6, 2012
\end{itemize}
ELSA is responsible for technical, organizational, and pedagogical support of Moodle.~\cite{elsa} 
However, the actual development on Moodle is outsourced to the different IT departments on the university. 
ELSA receives all feature requests and is responsible for providing support to the departments of the university, therefore they have keen knowledge about the problems and shortcomings of Moodle. 
We are only interested in improving Moodle in relation to the Aalborg PBL model, therefore we have chosen five subjects of those ELSA proposed that we deem relevant to this project:

\subsubsection{Automation of input data}
A course is automatically generated from a template. 
This empty course now needs to be filled with content, which requires manual work. 
This is primarily done by the administrative personnel and the lecturers. 
ELSA would like that this could be generated automatically based on central data. 
ELSA would prefer that new entities introduced by us were also generated automatically.

\subsubsection{Maintenance of data} Courses are, as mentioned above, maintained by the administrative personnel and the lecturers.
ELSA mentions that we should consider who should maintain the data introduced by features implemented by us.


\subsubsection{Overview of data} When a person is enrolled to several courses, the task of finding and entering the course page becomes difficult. 
ELSA would like a function in Moodle that renders a simpler overview, which eases the task of finding the wanted item.
We need to ensure that similar problems do not occur in any features we develop.

\subsubsection{Sharing of data} The ability to share files and relevant material is a central concern when engaged in project group work.
ELSA would like that features were more general such that they could be used in different contexts.
This could be the ability to share quiz question between courses. 

\subsubsection{Archiving} ELSA currently archives all courses and the related material. 
The archiving is done by closing all courses for enrollment and then copying all data from the Moodle installation. 
This approach does not work well, since students complain that they cannot enroll or unenroll from courses during the backup process. 

\subsection{MPBL}
\label{sub:mpblInterview}
\begin{itemize}
	\item Type: Informal meeting
	\item Participants: Jette Egelund Holgaard~\cite{jette}, Morten Mathiasen Andersen~\cite{morten}, and us
	\item Date: 08-02-12
\end{itemize}
The MPBL department defines themselves as:
``Master in Problem Based Learning is a fully online and highly interactive e-Learning programme for faculty staff at institutions who want to change to Problem Based and Project Based Learning (PBL)''.~\cite{mpbl}
At the department of MPBL they educate staff from other educational institutions in how to conduct PBL.
Moodle and other communication technologies are used in the education process. 
The duration of the online education is two and a half years.
To gain knowledge about how Moodle is used in professional and PBL context we conducted a meeting and discussed how Moodle can be improved to better support PBL. 
We chose to consider four of the subjects MPBL suggested:

\subsubsection{Joining of tools} MPBL explains that they are using Moodle as a part of their education but it does not support any functionality that aids PBL. 
To facilitate the online education other technologies such as Skype~\cite{skype} and Adobe Connect~\cite{adobe} are used. 
MPBL would like that the necessary tools are available in Moodle to avoid using several external tools.     

\subsubsection{Nearness} When working as a group it is an advantage to be in the same room.
When that is impossible, which is often the case for MPBL, then tools must create the feeling of nearness. 
A virtual meeting place could create the wanted feeling of nearness and thus the foundation of working as a group is set.

\subsubsection{Collaboration} Management and sharing of documents between the members of a group is a common issue. 
There is a need in Moodle to better handle sharing of documents. 
The ability to comment on documents is a common way to give feedback on other's work. 
Currently document sharing is conducted via emails, Dropbox~\cite{dropbox}, and Google docs~\cite{googledocs}.

\subsubsection{Planning} When group members are geographically separated it is essential to coordinate and plan.
MPBL would like that Moodle contains a tool to plan the collaboration process.