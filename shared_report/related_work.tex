\section{Related work}
We are not the first to undertake the task of implementing support for PBL and group-oriented work in an \textit{LMS}. 
As Moodle is not the only LMS, it is relevant to examine other LMSs to gain an understanding of how they typically are structured. 
Similarly they might provide inspiration for developing a new system.
The software that students from last year worked on, was of the same topic, and it is equally relevant to examine their results and extract key points.
There exist several plugins to Moodle that enhance it in respect to PBL. We will compare the most promising of these plugins and conclude if they can be used.

\subsection{Related Research}
\todo{Ud fra denne rapport: $http://vbn.aau.dk/files/62455944/Facilitating_Adoption_In_PBL_Activities.pdf$, kan vi læse hvilke værktøjer de studerende bruger(se tabel 3). List nogle af de commen activities som de studerende fortager sig. Disse activiteies kan bruges til at at argumentere for hvilke værktøjer de søger i et LMS}


\subsection{Learning Management Systems}
In this section we will examine a number of existing LMSs.
The systems that will be examined are \emph{SharePointLMS}, \emph{Litmos} and \emph{Mahara}.
We will examine how they handle students, courses and the concept of groups.
The objective is to gain an understanding of how an extension to Moodle could be structured, and which tools might be relevant to include in the system.

\subsubsection{SharePointLMS}
SharePointLMS \citep{sharepointlms} is an LMS based on the Microsoft SharePoint platform. 
It offers the basic functionality of an LMS, such as course management, student assessment tools such as quizzes and course certificates, conference tools, document sharing etc.

In terms of PBL, SharePointLMS offers some relevant features, such as creation of groups.
Within a group it is possible to enable features such as a chat, an internal mail, a calendar, and an online conference tool for meetings, which could be useful when working in a PBL context.
The only drawback is that groups cannot be created as independent entities in the system, but have to be created in the context of a course.
This does not fit well into the Aalborg PBL Model, where there is a clear distinction between courses projects.

\subsubsection{Litmos}
Litmos \citep{litmos} is a lightweight LMS with a focus on being easy to use and set up.
Its main features are creation of courses including multimedia content such as audio, assessment of students and surveys to gain feedback on courses.

In contrast to SharePointLMS, Litmos supports creation of groups as independent entities and even creation of groups within a group.
A group can be assigned to a course, which could be utilized to model the way that courses and projects are structured at AAU.
This would be achieved by creating one group containing all students on a given semester, and then assigning this group to the relevant courses.
Within this group a number of sub-groups would be created, representing the project groups.
Litmos has a built-in mailing system and is also integrated with Skype, which could be used to communicate within the groups.

\subsubsection{Mahara}
Mahara \citep{mahara} is technically not an LMS, but a \emph{Personal Learning Environment} (PLE), meaning that it is more learner-centered, as opposed to LMSs, which are typically more institution-centered.
However, Mahara still has some features which are relevant to PBL.

Mahara aims to be an online portal where students can share their work and be a member of communities within their area of interest.
It does not, however, come with a built-in calendar, which makes planning of projects an issue when just using Mahara.
Mahare compensates for this by providing a single-sign on bridge to Moodle, allowing users to access their Moodle accounts directly from Mahara without signing in again, and vice versa.

The aspect of Mahara relevant for PBL is its social networking feature, which allows users to maintain a list of friends and create groups.
Within a group it is possible to create a private forum as well as share files.

Overall Mahara provides a good platform for communicating within project groups, but it lacks support for planning and coordinating the projects.

\subsubsection{Comparison}
None of the examined LMSs provide complete support for PBL.
However, they do provide a variety of features, that if combined, would provide the functionality one would expect a PBL-oriented LMS to have.

The group structure found in Litmos could easily be combined with the features found in SharePoint to create a portal where students could organize their group work.
The functionalities typically provided in LMSs appear to be mostly forums, internal communication and planning tools.
None of the examined systems consider the aspect of communicating with a supervisor.

\subsection{Previous work}
The work done on last years 6$^{th}$ semester will be examined with a focus on the method used and the requirements for developed the system. 
The examined reports are: AAU E-Learning, E-LMS (1), E-Learning Management System - Implementing Customizable Schedules, and E-LMS (2).\todo{insert kilder til hver raport}


\subsubsection{Development Method}
The development method used by the groups were decided within each individual project group -- therefore no common method was employed in the multi-project.
The groups' collaboration was a meeting at least every second week with a representative from each group present, during which each group presented their progress since the last meeting.
Every month there would	e.g. be a meeting where all the students and the supervisors were present and the current status of the system was presented.
Their experience with this approach was that it was very difficult to coordinate the implementation of the system, as every group used their own development method. 
One group used a traditional approach, while others used a more agile approach.
This caused problems as the agile groups would want to do an integration, but since the traditional group had only just finished their analysis and design they were unable to.

The idea of having at least bi-weekly meetings seems good, but the development method should be modified so the meetings are not just status meeting, but are used for e.g. coordinating the implementation.
Having more frequents meetings is also a good idea, as having a common development method will make coordination more possible and relevant.

The reports from previous semesters clearly recommend the use of a shared development method, as otherwise coordination between the groups becomes too complicated. Similarly the different groups should actively try to coordinate and use each other instead of just working on their own and then communicate what they've achieved. 
This could be achieved by asking other groups for specific features at a coordination meetings and not assume they will be done eventually.


\subsection{Moodle Plugins}
\todo{skriv et afsnit om nogle pbl moodle plugins. Sammenlign dem og konkluder om det er noget vi kan bruge.}
\todo{Måske kunne der også stå noget om moodles groups, og hvorfor de ikke er brugbare til det vi skal.}
%\subsubsection{System analysis}
%The system developed last semester was an E-Learning system build from scratch. Therefore a lot of the focus was on administration and teaching in general. The system developed in %this project is an extension to Moodle which already has this functionality. Therefore most of the analysis part of these reports in irrelevant to this project. 



