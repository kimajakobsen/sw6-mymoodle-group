\section{Initial Analysis}
\begin{comment}
In the preliminary phase of working with Electronic-learning (E-learning) it is important for us to ensure that we are in fact working with problems that are relevant in the E-learning and Problem Based Learning (PBL) fields of study, as we are aiming at improving the overall way that the chosen E-learning environment at Aalborg University (AAU), namely Moodle, supports PBL. Because AAU is renowned for its implementation of PBL, we do not have to look very far to find some of the leading experts in this field.
\end{comment}



After interviews with the management team of Moodle at AAU (ELSA) and the PBL section of the institute of planning at AAU, it has become clear to us that Moodle in its current implementation only supports PBL vaguely, if it supports it all. The institute of planning, who teaches PBL through Internet classes based on Moodle, has the problem that the platform they use in their PBL-education does not support PBL, which is the thing they are trying to teach others how to implement. The institute of planning is especially concerned with the lack of an environment in which their Internet-students could connect as a group, because the division of students into groups is one of the fundamental ideas of PBL(KILDE!?!). After interviewing these experts, we interviewed some of the students of AAU, in order to try to understand their use of Moodle, and how to make it a better experience. They told the same story; the lack of group rooms or other places to meet and work together is a pressing concern of most of the students of the faculties of social sciences and the faculty of humanities. Based on these discoveries we have come to the conclusion that it would be useful to have online group rooms, to accommodate the needs of students without available group rooms. This will make it easier to switch between different rooms each day and work together, even if the collaborators of the project are not able to meet physically.
In order to create the group rooms we discovered the following four feasible goals within our limited time-scope:
A virtual blackboard able to save drawings
An easy to use timeline tool to display various deadlines and events
An easy and consistent way for students and supervisors to communicate
Tools allowing the administrative personnel to support all of this new functionality

Because all of this functionality has to be developed on an already existing platform, and not as part of a platform, we have chosen to divide these tasks vertically, as illustrated in Figure \ref{fig:target_group}.


\begin{figure}
	\centering
		\includegraphics[width=\textwidth]{\sharedReport target_group.pdf}
	\caption{System decomposition}
	\label{fig:target_group}
\end{figure}
