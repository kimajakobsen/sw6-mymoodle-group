\section{E-Learning}
The term e-learning covers all forms of electronically supported learning. E-learning is often associated with distance learning and out-of-classroom teaching, but can also be used to support traditional in-classroom teaching. In short E-learning is defined as learning that is facilitated and supported via \emph{ICT} (Information and Communications Technology). The coorporation between the teachers and students and between the students themselves can similarly be partially or completely conducted via ICT~\citep{def_e-learning1}~\citep{def_e-learning2}.

E-learning is growing in popularity, it was e.g. noted in 2007 that 20\% of all higher education students in the U.S. were taking at least one course online, and that does not include the courses which makes use of some form of e-learning.	At \emph{AAU} (Aalborg University) e-learning is employed as well in a variety of forms. There are fully online courses taught by e.g. The Institute for \emph{PBL} (Problem Based Learning), and regular courses that make use of e.g. online quizes as a method of teaching\todo{Bad sentence, revise later}. 

Moodle is currently the primary e-learning platform at AAU. Its main purpose is to allow lecturers to share course-relevant material with students and to serve as a calender service containing dates of lectures, meetings etc. The problem is that Moodle does not support PBL, which is the learning method used at AAU.

\subsection{The Aalborg PBL Model}
The Aalborg PBL model is term coined to describe Aalborg University's problem and project based learning model. It originates from the philosophy the university's faculty had at its creation. They were interested in giving the students an active role in obtaining knowledge, as opposed to the lecture based learning used at many universities. Furthermore they desired to give the faculty an more active role in the students experience than the lecture setting provided. From this the Aalborg PBL Model was developed.

The Aalborg PBL Model has six terms which describes the basic concept of the model:

\begin{itemize}
	\item \textbf{Problem} - A written formulation of a problem which can be theoretical, practical etc. It is sparked by the students curiosty. The problem serves as the starting point for the student's learning and puts into a concrete context. A problem must be exemplary and can contain interdisciplinary approaches	in both the analysis and solving phase.
	\item \textbf{Project} - A task which requires an analysis of a new and complex problem. It must be planned and managed, as it might potentially affect people's surroundings, organization, knowledge and attitude to life, and it must be completed on time. Projects are diverse in regards to scope and definition.
	\item \textbf{Exemplarity} - The problem a project is concerned with must be exemplary, that is it must relateable to a specific practical, scientific or technical domain.
	\item \textbf{Team} - A team is a group working together on a project. Their coorporation on the different aspects of a succesfully completed project is essential to the approach to learning.
	\item \textbf{Supervisor} - A role ususally held by faculty members, the supervisor serves a resource for a group of students working on a project to rely on for help and feedback. The supervisor is only assigned to a particular group for the duration of the project. 
	\item \textbf{Courses} - Courses are given to the students which fit within the theme for a semester. Earlier courses were divided into project courses and study courses, but the new study regulation has removed this distinction. Courses provide knowledge to the students either relevant for their project or for their field.
\end{itemize}

The Aalborg PBL Model is build up of 6 core principles, which outline the students learning proces.
\begin{itemize}
	\item \textbf{Problem orientation} - A problem relevant to the students' field serve as the basis for their learning proces
	\item \textbf{Project organization} - The project is the medium through which the students address the problem and achieve the knowledge outlined by the curriculum
	\item \textbf{Integration of theory and practice} - The curriculum and staff at the university are responsible for teaching the students to connect their project work to broader theoretical knowledge
	\item \textbf{Participant direction} - Students define their problem and make decisions relevant for the completion of the project themselves
	\item \textbf{Team-based approach} - Most of the students project related work is conducted in groups of three or more students
	\item \textbf{Collaboration and feedback} - Peer and supervisor critique is used continously throughout a project to improve the students work. The aim is the students gain the skills of collaboration, feedback and reflection from following the PBL model. 
\end{itemize}

These concepts and principles make up the core of the learning method used at AAU. Most semesters are centered around projects whose scope is set within a given semester theme. The students are given lectures which both supports their project, but also independent courses teaching students material within their field. The faculty at AAU take an active role in the students learning proces by serving as supervisors on projects. Similarly students are themselves responsible for obtaining knowledge in their projects, putting an emphasis on self-learning. This section is based on~\citep{Barge10}.

\section{Moodle}
Moodle (\emph{Modular Object-Oriented Dynamic Learning Environment}) \citep{moodle} is an e-learning platform for producing dynamic web sites for courses. Moodle was originally developed by Martin Dougiamas in 2002 and is released under an open source license (GPLv3+) \citep{gpl}. It is currently maintained by a group of core developers. Due to its modular design, the functionality of Moodle can be extended with plugins developed by the Moodle community.

Moodle is build up around the concept of courses, and most activities in Moodle are centered arround them. Courses can be divided into categories, and topics, recources, activites and block can be added to them. Topics is an integrated part of courses, and resources are e.g. links to external sources or uploaded files. Both these elements are an integrated part of courses. Activities and blocks are plugins which can be added to a course to provide a functionality. They differentiated by how they are integrated in Moodle. An activity is added by placing a link on e.g. a topic to another page where the activites functionality lies. A block is shown visually on the course page in e.g. the right sidebar \todo{Add picture of a Moodle course page for reference and source: http://www.slideshare.net/mark.drechsler/moodle-structural-overview}.

However, as Moodle is build up around courses it does not provide any facilities to support the way that students work on projects at AAU. The students collaborate in groups in order to solve realistic problems, which is also known as \emph{problem-based learning} (PBL), or more specific, the \emph{Aalborg PBL Model} \citep{pbl}. This need serves as the motivation for the following problem statement.

\section{Problem Statement}
Moodle does not currently provide any facilities to support PBL. How can Moodle be extended to support PBL? Which features would be required to do so?