\chapter{Introduction}

\section{E-Learning}
The term e-learning covers all forms of electronically supported learning. E-learning is often associated with distance learning and out-of-classroom teaching, but can also be used to support traditional in-classroom teaching.

Moodle is currently the primary e-learning platform at AAU. Its main purpose is to allow lecturers to share course-relevant material with students and to serve as a calender service containing dates of lectures, meetings etc.

\section{Moodle}
Moodle (\emph{Modular Object-Oriented Dynamic Learning Environment}) \citep{moodle} is an e-learning platform for producing dynamic web sites for courses. Moodle was originally developed by Martin Dougiamas in 2002 and is released under an open source license (GPLv3+) \citep{gpl}. It is currently maintained by a group of core developers. Due to its modular design, the functionality of Moodle can be extended with plugins developed by the Moodle community.

However, Moodle does not provide any facilities to support the way that students work on projects at AAU. The students collaborate in groups in order to solve realistic problems, which is also known as \emph{problem-based learning} (PBL), or more specific, the \emph{Aalborg PBL Model} \citep{pbl}. This need serves as the motivation for the following problem statement.

\section{Problem Statement}
Moodle does not currently provide any facilities to support PBL. Could Moodle be extended to support PBL? Which features would be required to do so?