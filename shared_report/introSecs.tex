\section{E-Learning}
The term e-learning covers all forms of electronically supported learning. E-learning is often associated with distance learning and out-of-classroom teaching, but can also be used to support traditional in-classroom teaching.

Moodle is currently the primary e-learning platform at AAU. Its main purpose is to allow lecturers to share course-relevant material with students and to serve as a calender service containing dates of lectures, meetings etc.

\subsection{The Aalborg PBL Model}
The Aalborg PBL model is term coined to describe Aalborg University's problem and project based learning model. It originates from the philosophy the faculty at the university. They were interested in giving the students an active role in obtaining knowledge as opposed to the lecture based learning used at many universities. Furthermore they desired to give the faculty an more active role in the students experience than the lecture setting provided. The result was The Aalborg PBL Model.

The Aalborg PBL Model has six terms which describes the basic concept of the model:

\begin{itemize}
	\item \textbf{Problem} - A written formulation of a problem which can be theoretical, practical etc. It is sparked by the students curiosty. The problem serves as the starting point for the student's learning and puts into a concrete context. A problem must be exemplary and can contain interdisciplinary approaches	in both the analysis and solving phase.
	\item \textbf{Project} - A task which requires an analysis of a new and complex problem. It must be planned and managed, as it might potentially affect people's surroundings, organization, knowledge and attitude to life, and it must be completed on time. Projects are diverse in regards to scope and definition.
	\item \textbf{Exemplarity} - The problem a project is concerned with must be exemplary, that is it must relateable to a specific practical, scientific or technical domain.
	\item \textbf{Team} - A team is a group working together on a project. Their coorporation on the different aspects of a succesfully completed project is essential to the approach to learning.
	\item \textbf{Supervisor} - A role ususally held by faculty members, the supervisor serves a resource for a group of students working on a project to rely on for help and feedback. The supervisor is only assigned to a particular group for the duration of the project. 
	\item \textbf{Courses} - Courses are given to the students which fit within the theme for a semester. Earlier courses were divided into project courses and study courses, but the new study regulation has removed this distinction. Courses provide knowledge to the students either relevant for their project or for their field.
\end{itemize}

The Aalborg PBL Model is build up of 6 core principles, which outline the students learning proces.
\begin{itemize}
	\item \textbf{Problem orientation} - A problem relevant to the students' field serve as the basis for their learning proces
	\item \textbf{Project organization} - The project is the medium through which the students address the problem and achieve the knowledge outlined by the curriculum
	\item \textbf{Integration of theory and practice} - The curriculum and staff at the university are responsible for teaching the students to connect their project work to broader theoretical knowledge
	\item \textbf{Participant direction} - Students define their problem and make decisions relevant for the completion of the project themselves
	\item \textbf{Team-based approach} - Most of the students project related work is conducted in groups of three or more students
	\item \textbf{Collaboration and feedback} - Peer and supervisor critique is used continously throughout a project to improve the students work. The aim is the students gain the skills of collaboration, feedback and reflection from following the PBL model. 
\end{itemize}
\section{Moodle}
Moodle (\emph{Modular Object-Oriented Dynamic Learning Environment}) \citep{moodle} is an e-learning platform for producing dynamic web sites for courses. Moodle was originally developed by Martin Dougiamas in 2002 and is released under an open source license (GPLv3+) \citep{gpl}. It is currently maintained by a group of core developers. Due to its modular design, the functionality of Moodle can be extended with plugins developed by the Moodle community.

However, Moodle does not provide any facilities to support the way that students work on projects at AAU. The students collaborate in groups in order to solve realistic problems, which is also known as \emph{problem-based learning} (PBL), or more specific, the \emph{Aalborg PBL Model} \citep{pbl}. This need serves as the motivation for the following problem statement.

\section{Problem Statement}
Moodle does not currently provide any facilities to support PBL. Could Moodle be extended to support PBL? Which features would be required to do so?