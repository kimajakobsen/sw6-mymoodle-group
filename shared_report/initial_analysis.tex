\section{Initial Analysis}
\label{sec:initialAnalysis}
To determine how to improve Moodle to support MPBL we conducted two informal discussions with the ELSA department\cite{elsa}\todo{Hvis dette er det første sted elsa bliver nævnt så skal forkortelsen skrives ud, dette gælder også for mpbl} and the MPBL department\cite{mpbl}. This section will account for the results of the discussion and based on that how we decides to solve the problem explained in conceptual terms. A central discussion topic is introducing project groups\ref{} \todo{ref til der hvor der står noget om gruppearbejde, det står længere oppe i introen.} into Moodle and how they are to be used.     

\subsection{Discussion with expert users}  
\label{sub:expertUsers} 
We conducted a meeting with the two departments mentioned above. On both meeting we discussed which improvements we could make to Moodle in regards to PBL.  

\subsubsection{ELSA}
\begin{itemize}
	\item Type: Informal meeting
	\item Participants:Lillian Buus, Marie Glasemann, Mads Peter Bach, and the E-Learning group \todo{ikke sikket mads deltog i dette møde.} \todo{hvordan laver man kilder på personer?}
	\item Date: 06-02-12
\end{itemize}
ELSA is responsible for technical, organizational, and pedagogical support of Moodle\cite{elsa}. 
However, the actual development on Moodle is outsourced to the different IT departments on the university. 
ELSA receives all feature requests and is responsible for providing support to the departments of the university, therefore they have keen knowledge about the problems and shortcomings of Moodle. 
We are only interested in improving Moodle in regard to PBL, therefore, we have chosen four subject that we deem are relevant to this project:

\paragraph{Automation of input data} A template of a course are autocratically generated, this empty course now need to be populated with content this requires manual work. 
The population of a course are primarily done by the administrative personal and the lectures. 
ELSA wishes that this could be generated automatically based on central data, new features such as project groups should also be generated automatically.     
\paragraph{Maintenance of data} Courses are, as mentioned above, maintained by the administrative personal and the lectures, likewise should group room be maintainable by the students. 
\paragraph{Overview of data} When a person is enrolled to several courses, the task of finding and entering the course page becomes difficult. 
ELSA wishes a new system\todo{kald det noget andet end et nyt system.} which renders a greater overview and ease the task of find the wanted item. 
This problem could also occur if a person is a member of several project groups.
\paragraph{Sharing of data} The ability to share files and relevant material is a central concern when engaged in project group work. ELSA furthermore wish to have more modular tools that are able to interact and share data.
This could be the ability to share quiz question between courses. 


\subsubsection{MPBL}
\begin{itemize}
	\item Type: Informal meeting
	\item Participants:Jette Egelund Holdgaard, Morten Mathiasen Andersen, and the E-Learning group
	\item Date: 08-02-12
\end{itemize}
``Master in Problem Based Learning is a fully online and highly interactive e-Learning programme for faculty staff at institutions who want to change to Problem Based and Project Based Learning (PBL) ''\cite{mpbl}.
At the department of MPBL they educate staff from other educational institutions in how to conduct PBL \ref{}\todo{det står længere oppe}. Moodle and other communications technologies are used in the education process. 
The duration of the online education is 2$\frac{1}{2}$ year.
To gain knowledge about how Moodle is used in professional and PBL context we conducted a meeting and discussed how Moodle can be improved to better support PBL. Again we choose four subject that we deem covers the wishes expressed by MPBL:

\paragraph{Joining of tools} MPBL explains that they are using Moodle as a part of their education but it does not support any functionality that aid PBL. 
To facilitate the online education other technologies such as Skype\cite{skype} and Adobe Connect\cite{adobe} are used. 
MPBL wish that the necessary tools are available in Moodle to avoid using several external tools.     

\paragraph{Nearness} When working as a group it is an advantage to be in proximity of each other. When that is impossible, which is often the case for MPBL, then tools must create the feeling of nearness. 
A virtual meeting place made disposable could create the wanted feeling of nearness and thus be the foundation of working as a group is set. 

\paragraph{Collaboration} Management and sharing of document between the members of a group is a common issue. 
There is a need in Moodle to better handle sharing of documents. The ability to comment on documents are a common way to give feedback on others work. 
Currently document sharing are conducted via emails, Dropbox\cite{dropbox} or Google docs\cite{googledocs}.


\paragraph{Planning} When group members are geographically 






virtuelt mødested (bruger skype lige nu), føgelsen af nærhed, mere projekt arbejde, mulighed for at kommentere på hinanden ting, mødeplanlægning, google docs funktionalitet , process styreing / planlægning, 

\subsection{Virtual Grouproom}


\begin{comment}
In the preliminary phase of working with Electronic-learning (E-learning) it is important for us to ensure that we are in fact working with problems that are relevant in the E-learning and Problem Based Learning (PBL) fields of study, as we are aiming at improving the overall way that the chosen E-learning environment at Aalborg University (AAU), namely Moodle, supports PBL. 
Because AAU is renowned for its implementation of PBL, we do not have to look very far to find some of the leading experts in this field.
\end{comment}
\begin{comment}
To optian domain knowlage of how moodle and Aapbl works 
Domain knowlage
In the initial phase of this project we conducted two interviews with xxx from ELSA\cite{x} and yyy from MPBL. 

interviews
*ELSA
*MPBL

*Baseret på de ting de fortalte ->
*Kommunication
* 
*Fildeling

Opdeling i de 4 områder

*akritektur
\end{comment}

\begin{comment}
After interviews with the management team of Moodle at AAU (ELSA) and the PBL section of the institute of planning at AAU, it has become clear to us that Moodle in its current implementation only supports PBL vaguely, if it supports it all. The institute of planning, who teaches PBL through Internet classes based on Moodle, has the problem that the platform they use in their PBL-education does not support PBL, which is the thing they are trying to teach others how to implement. The institute of planning is especially concerned with the lack of an environment in which their Internet-students could connect as a group, because the division of students into groups is one of the fundamental ideas of PBL(KILDE!?!). After interviewing these experts, we interviewed some of the students of AAU, in order to try to understand their use of Moodle, and how to make it a better experience. They told the same story; the lack of group rooms or other places to meet and work together is a pressing concern of most of the students of the faculties of social sciences and the faculty of humanities. Based on these discoveries we have come to the conclusion that it would be useful to have online group rooms, to accommodate the needs of students without available group rooms. This will make it easier to switch between different rooms each day and work together, even if the collaborators of the project are not able to meet physically.
In order to create the group rooms we discovered the following four feasible goals within our limited time-scope:
A virtual blackboard able to save drawings
An easy to use timeline tool to display various deadlines and events
An easy and consistent way for students and supervisors to communicate
Tools allowing the administrative personnel to support all of this new functionality

Because all of this functionality has to be developed on an already existing platform, and not as part of a platform, we have chosen to divide these tasks vertically, as illustrated in Figure \ref{fig:target_group}.


\begin{figure}
	\centering
		\includegraphics[width=\textwidth]{\sharedReport target_group.pdf}
	\caption{System decomposition}
	\label{fig:target_group}
\end{figure}
\end{comment}