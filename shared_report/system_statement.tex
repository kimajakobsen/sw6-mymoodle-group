\section{System Definition}
\todo{beskrivelse af fælles system system som vi vil lave. Hver gruppe skal ahve deres egen system statement, der definere hvad gruppen konkret arbejder med.}
Based on our problem definition, the related work we have gathered, and our initial analysis, we define \system as follows:

\sharedInput{system_definition}











\subsection{Decomposing \system}
\label{sub:decomposingSys}
The project that we are undertaking is rather large.
This leads us to decompose into smaller sub-projects.
This also makes sense with respect to the context that the project is being conducted, namely an educational context where we are supposed to work in one large group divided into smaller sub-groups.
Each sub-group will have its own sub-project which goal is to make a given sub-system, integrate it with the other sub-systems, and write a report to document the process.

The final system, as defined above, should enable PBL in Moodle.
We will divide the system into sub-systems in accordance with the core principles of Aalborg PBL as defined in Section \ref{sub:aaupbl}.
The sub-systems that \system is divided into are:
\begin{itemize}
	\item \textbf{Project organization --} The students can plan the course of their projects themselves.
	This sub-system is responsible for allowing students to make a schedule of a project and assign tasks to students.
	\item \textbf{Participant direction --} During a project, them participants must be able to communicate and thereby direct the course of the project.
	The communication must be persistent, such that the participants can use their communication as documentation should it be needed.
	\item \textbf{Team-based approach --} At a given project there will be associated a team of students.
	Only the students associated with a project should be allowed to contribute directly.
	The participants and their privileges must be managed in this sub-system.
	\item \textbf{Collaboration and feedback --} The participants must be able to communicate and receive feedback from their supervisor.
	Students should be able to give feedback to projects, provided that the participants of the project allows it.\todo{Måske skal vi strege denne her. Ellers kan man argumentere for at det er noget der bør komme i næste forløb}
\end{itemize}

It should be noted that only four of the six core principles have a corresponding 










